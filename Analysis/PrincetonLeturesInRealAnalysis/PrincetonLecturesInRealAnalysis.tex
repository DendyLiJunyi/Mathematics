% Created 2024-11-07 Thu 11:45
% Intended LaTeX compiler: xelatex
\documentclass[11pt]{article}
\usepackage{graphicx}
\usepackage{longtable}
\usepackage{wrapfig}
\usepackage{rotating}
\usepackage[normalem]{ulem}
\usepackage{capt-of}
\usepackage{hyperref}
\input{~/Preamble/preamble.tex}
\author{Elias M. Stein \\
Rami Shakarchi Notes taken by Dendy Li}
\date{\today}
\title{\elzevier Princeton Lectures In Real Analysis}
\hypersetup{
 pdfauthor={Elias M. Stein \\
Rami Shakarchi Notes taken by Dendy Li},
 pdftitle={\elzevier Princeton Lectures In Real Analysis},
 pdfkeywords={},
 pdfsubject={},
 pdfcreator={Emacs 29.3 (Org mode 9.7.11)}, 
 pdflang={English}}
\begin{document}

\maketitle
\tableofcontents

\section{Lebesgue Dominated Convergence Theorem}
\label{sec:org82b0a70}
\begin{definition}[Dominated Convergence Theorem]
If we have a sequence of measurable functions \(\{f_n\}\) converges to \(f\) a.e., if \(\lvert f_n \rvert \leq g\) this implies the absolute value of the difference of \(f_n - f\) tends to \(0\). As a consequence the integral of \(f_n\) tends to the integral of \(f\).
\end{definition}
\subsection{Almost everywhere}
\label{sec:orgd49cad9}
\begin{definition}[Almost everywhere equal]
\(f,g\) defined on a set \(E\) are equal almost everywhere, we write
\[
f(x) = g(x) \ \text{a.e. } x\in E,
\]
If the measure of the set which consists of the point \(p\) where \(f(p) \neq g(p)\) has measure zero.
\end{definition}

\begin{definition}[Almost everywhere convergence]
We say a sequence of functions \(\{f_n\}\) converges almost everywhere to \(f\) on set \(E\) if:
\[
\{x\in E | \{f_n(x)\} \nrightarrow f(x)\}
\]
has measure \(0\).
\end{definition}
\textbf{Remark:}
\begin{itemize}
\item \textbf{a.e. = a.s. = p.a.e(Pointwise almost everywhere)}
\item \textbf{Pointwise converges doesn't implies the measure, it just says that on every point of the domain we care, the corresponding sequence converges.}
\end{itemize}

\begin{definition}[Measurable Function]
A function \(f\) on \(\R^d\) is measurable if for every \(a \in \R\):
\[
\{x \in \R^d \vert f(x) < a\}
\]
is measurable.
\end{definition}

To find some intuitive examples to this definition of measurable funcitons, characteristic functions and step functions(simple functions) will be our choices.
\subsection{Integration}
\label{sec:org385b031}
\section{Littlewood's Three Principle}
\label{sec:org31385d4}
\begin{theorem}[Littlewood's principle]
\leavevmode
\begin{itemize}
\item Every measurable function is almost continuous(Lusin);
\item Every measurable set is almost a finite union of intervals;
\item Every converges sequence is almost uniformly converges(Egorov Theorem).
\end{itemize}
\end{theorem}
\section{Egorov Theorem}
\label{sec:org47ffd19}
\begin{definition}[Egorov Theorem]
Suppose we have a sequence of measurable functions defined on set \(E\) with \(m(E)\) is finite. Assume the sequence of measurable functions converges to some function \(f\) a.e. on \(E\). Given a \(\epsilon\), we can find a closed subset \(A_{\epsilon}\) of \(E\) s.t. \(A_{\epsilon}\) is differ from \(E\) with a set which has measure \(\epsilon\) and converges becomes uniform converges on this closed set.
\end{definition}
\subsection{Measurable Functions}
\label{sec:org69f2e12}
The definition of measurable functions is a progressive definition, we first look at the simple function.

\begin{definition}[Finite-value function is measurable]
A finite-value function is measurable iff preimage of any openset \(O\) is measurable iff preimage of any closed set \(F\) is measurable.
\end{definition}
\textbf{Remark: We use \(F\) to declare the closed sets since fermé.}
\end{document}
